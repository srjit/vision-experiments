
\documentclass{article}

%\usepackage[utf8]{inputenc}
%\usepackage[a4paper, total={7in, 8in}]{geometry}
\usepackage[left=15mm,top=26mm,right=8mm,bottom=15mm]{geometry}

\usepackage{spverbatim}
\usepackage{minted}
\usepackage{amssymb}
\usepackage{bm}
\usepackage{graphicx}
\usepackage{amsmath}
\usepackage{enumerate}
%\setlength{\voffset}{-0.75in}
%\setlength{\headsep}{-10pt}


\title{EECE 5639- Homework 4}
\author{Sreejith Sreekumar: 001277209}
\date{\today}
\begin{document}

\maketitle

\section*{Question 1}
\begin{enumerate}[(a)]

\item   Planar projective transformation in homogeneous coordinates:
\begin{gather}
 \begin{bmatrix} x \\ y \\ 1  \end{bmatrix}
 =
\begin{bmatrix} 
h_{11} & h_{12} & h_{13}\\
h_{21} & h_{22} & h_{23}\\
h_{31} & h_{32} & h_{33}\\
\end{bmatrix}
 \begin{bmatrix} p \\ q \\ 1  \end{bmatrix}
\end{gather}

\item There are 8 degrees of freedom for this transformation.
\item 4 point correspondences are required for this transformation.
\item Yes. If more points are available, the caliberation necessary for improving the
  accuracy can be done more precisely (eg. Using linear least squares).
\item Collinearity, order of contact, Tangent discontinuities and cusps,
  cross ratio of 4 collinear points etc. are some invarients in planar projective transformation.
\item  Ratio of lengths on collinear or parallel lines, ratio of areas etc. are some invarients.
\item  Both sets have their vanishing point at infinity. In this case the building will be fronto-parallel. 
\end{enumerate}

\section*{Question 2}

\begin{minted}{Matlab}

% create the image
image = zeros(8,8);
image(6,1) = 1;
image(7,2) = 1;
image(8,2) = 1;
image(6,3) = 1;
image(2,2) = 2;
image(6,2) = 2;
image(2,4) = 2;
image(6,6) = 2;
image(4,7) = 2;
image(5,7) = 2;
image(6,8) = 2;
image(3,2) = 2;
image(2,3) = 4;
image(4,3) = 2;
image(6,7) = 4;


%create the patch
template = zeros(3,3);
template(1,1) = 1;
template(1,3) = 1;
template(2,2) = 1;
template(3,2) = 1;
template(1,2) = 2;

\end{minted}  

\begin{enumerate}[(a)]  
\item  Calculation of SSD \\
  \begin{minted}{Matlab}
% calculation of SSD
f = image;
g = template;
ssd = zeros(8,8);

for i = 2:7
    for j = 2: 7
           var  = (f(i-1, j-1) - g(1,1)).^2 + ...
                + (f(i-1, j) - g(1,2)).^2+ ...
                + (f(i-1, j+1) - g(1,3)).^2 + ...
                + (f(i, j-1) - g(2,1)).^2 + ...
                + (f(i,j) - g(2,2)).^2 + ...
                + (f(i,j+1) - g(2,3)).^2 + ...
                + (f(i+1,j-1) - g(3,1)).^2 + ...
                + (f(i+1,j) - g(3,2)).^2 + ...
                + (f(i+1,j+1) - g(3,3)).^2;
       ssd(i,j) = sum(sum(var));
    end
end

ssd(1,:) = 0;
ssd(:,1) = 0;
ssd(8,:) = 0;
ssd(:,8) = 0;

disp("SSD is:");
disp(ssd)

SSD is:
     0     0     0     0     0     0     0     0
     0    24    28    24    12     8     8     0
     0    16    12    16     8    12     8     0
     0     8     8    12     8    16     8     0
     0    10     7     9    12    28    20     0
     0     9    12     9    12    24    20     0
     0     0     7     7     8    12     8     0
     0     0     0     0     0     0     0     0
  \end{minted}

\item Calculation of correlation

  \begin{minted}{Matlab}
cross_correlation_score = imfilter(image,template);
cross_correlation_score(1,:) = 0;
cross_correlation_score(:,1) = 0;
cross_correlation_score(8,:) = 0;
cross_correlation_score(:,8) = 0;
disp("Cross Correlation score");
disp(cross_correlation_score);

Cross Correlation score

     0     0     0     0     0     0     0     0
     0     4     4     2     0     0     0     0
     0    10    14     8     2     0     2     0
     0     4     4     0     0     0     4     0
     0     4     5     2     0     4    10     0
     0     3     1     0     0     4     8     0
     0     8     4     1     2     8    12     0
     0     0     0     0     0     0     0     0
     
  \end{minted}

\item Calculation of Normalized Cross Correlation

 \begin{minted}{Matlab}
normalized_image = zeros(8,8);
normalized_template = zeros(3,3);

% normalizing the f matrix - f/||f||
for i=2:7
    for j=2:7
      denominator = f(i-1, j-1) .^2 + ...
                  + f(i-1, j) .^2 + ...
                  + f(i-1, j+1) .^2 + ...
                  + f(i, j-1) .^2 ...
                  + f(i,j) .^2 ...
                  + f(i,j+1) .^2 ...
                  + f(i+1,j-1) .^2 ...
                  + f(i+1,j) .^2 ...
                  + f(i+1,j+1) .^2;
       denominator = sqrt(denominator);
       if denominator == 0
        normalized_image(i,j) = 0;
       else
        normalized_image(i,j) = (f(i,j) / denominator);
       end
    end
end

% normalizing the g matrix - g/||g||
normalized_template = g./sqrt(sum(sum(g.^2)));
normalized_cross_correlation = imfilter(normalized_image, normalized_template);

normalized_cross_correlation(1,:) = 0;
normalized_cross_correlation(:,1) = 0;
normalized_cross_correlation(8,:) = 0;
normalized_cross_correlation(:,8) = 0;

disp("Normalized Cross Correlation");
disp(normalized_cross_correlation);


Normalized Cross Correlation
         0         0         0         0         0         0         0         0
         0    0.2887    0.2673    0.1581         0         0         0         0
         0    0.6682    0.8750    0.5774    0.3536         0    0.3536         0
         0    0.5000    0.5000         0         0         0    0.5000         0
         0    0.4472    0.5893    0.3162         0    0.2673    0.6250         0
         0    0.4009    0.1443         0         0    0.2887    0.5345         0
         0    1.0000    0.5345    0.3536    0.3536    0.6325    0.8660         0
         0         0         0         0         0         0         0         0

 \end{minted}

\end{enumerate}    

\end{document}


